\documentclass[BCOR=12mm,DIV11,titlepage,a4paper,oneside]{scrbook}

%Paket für deutsche Silbentrennung etc.
\usepackage{ngerman}

%Paket für Zeichenkodierung, entspricht UTF-8
\usepackage[utf8x]{inputenc}

%Paket für PostScript Schriftarten
\usepackage{pslatex}

%Paket das die Ausgabefonts definiert
\usepackage[T1]{fontenc}

%Paket für Sonderzeichen wie RightsReserved
\usepackage{textcomp}

%Euro-Symbol
\usepackage[right]{eurosym}

%Paket für das Einbinden von Grafiken über die figure-Umgebung
\usepackage{graphicx}

%Paket zum Ändern der Kopf- und Fußzeile
\usepackage{fancyhdr}
%Benutzt das Paket für eigenen Seitenstil
\pagestyle{fancy} 
%Erzeugt eine Linie in der Kopfzeile (lässt sich mit 0.0pt ausblenden)
\renewcommand*{\headrulewidth}{0.4pt} 
\fancyhf{}
\fancyhead[EC,OC]{\thepage}
% \fancyhead[EL]{\leftmark} 
% \fancyhead[OR]{\rightmark} 
% \fancyhead[ER,OL]{\thepage}
\renewcommand{\sectionmark}[1]{ 
\markboth{\thechapter{} #1}{\thechapter{} #1} 
}
%Ändert die Seitennummerierung beim Inhaltsverzeichnis mit eigenem Stil
\renewcommand*{\indexpagestyle}{fancy}
%Verhindert die Seitennummerierung auf den Part-Seiten
\renewcommand*{\partpagestyle}{empty}
%Ändert die Seitennummerierung bei Chapter mit eigenem Stil
\renewcommand*{\chapterpagestyle}{fancy}

%Abbildungsnummerierung ändern (abhängig von chapter, z.B. Abbildung 1.1)
\renewcommand*{\thefigure}{\thechapter.\arabic{figure}}
%Tabellennummerierung ändern (abhängig von chapter, z.B. Tabelle 1.1)
\renewcommand*{\thetable}{\thechapter.\arabic{table}}

%Paket, um ein Glossar/Abkürzungsverzeichnis anzulegen
\usepackage{nomencl}
\let\abbrev\nomenclature
%Der Name wird in Glossar geändert
\renewcommand{\nomname}{Glossar (optional)}
%Definiert die Aufteilung im Glossar zwischen Begriffen und Erläuterung
\setlength{\nomlabelwidth}{.25\hsize}
%Definiert die Punktelinien im Glossar
\renewcommand{\nomlabel}[1]{#1 \dotfill}
\setlength{\nomitemsep}{-\parsep}
%Veranlasst die Erstellung des Glossars
\makenomenclature

%Einrückungen nach Absätzen und Grafiken verhindern
\setlength{\parindent}{0pt}



%Verhindern, dass eine neue Seite für ein einzelnes Wort/Zeile verwendet wird
\clubpenalty = 10000 % schliesst Schusterjungen aus 
\widowpenalty = 10000 % schliesst Hurenkinder aus (keine Beleidigung, sondern wirklich ein Fachbegriff)

%Paket für ein deutsches Literaturverzeichnis
% \usepackage[authoryear,round]{natbib}
% \bibliographystyle{natdin}
% \setlength\bibhang{30pt}
\usepackage{bibgerm}

%Paket für die Verwendung von URLs durch den Befehl \url{}
\usepackage{url}

%Paket für Zeilenabstand (onehalfspace, singlespace)
\usepackage{setspace}

%Paket zur Erzeugung von Anführungszeichen durch \enquote{Text}
\usepackage[ngerman]{babel}
\usepackage[babel, german=quotes]{csquotes}

%Paket für farbigen Text
%black,white,green,red,blue,yellow,cyan,magenta
\usepackage{color}
%Farbige Tabellen
\usepackage{colortbl}

%Rotation von Gleitobjekten (Grafiken, Trabellen, etc.)
\usepackage{rotating}

%Rotation von einzelnen Seiten begin{landscape}
\usepackage{lscape}

%Paket für farbigen Hintergrund für Verbatim-Umgebung (Quelltext-Umgebung)
\usepackage{fancyvrb}
\usepackage{verbatim,moreverb}
%Grauton für Quelltext-Umgebung definieren 80% Grau
\definecolor{sourcegray}{gray}{.80}
%Paket für Quelltext-Umgebung
\usepackage{listings}
%Alternative Quelltext-Umgebung
%\lstset{numbers=left, 
%	numberstyle=\tiny, 
%	numbersep=5pt,
%	language=Java,
%	breaklines=true,
%	breakautoindent=true,
%	postbreak=\space,
%	tabsize=2,
%	frame=tlrb,
%	basicstyle=\ttfamily\footnotesize}

%Paket für Positionierung der Objekte ohne Float (Verwendungsbsp.: \begin{figure}[H])
%\usepackage{here}
%Alternatives Paket für here.sty
\usepackage{float}

%DRAFT als Wasserzeichen im Hintergrund
% \usepackage{draftwatermark} 
% \SetWatermarkAngle{60}
% \SetWatermarkScale{5.0}

%Für lange Tabellen
\usepackage{longtable,array,supertabular}

%Für rowspan in Tabellen
\usepackage{multirow}

%Behält die Schriftgröße der Überschrift normal, wenn z.B. die Schriftgröße in einer Tabelle verändert wird
\addtokomafont{caption}{\normalsize} 

%Paket um PDF Seiten einzubinden
\usepackage{pdfpages}

%Paket zur Erzeugung von Hyperrefs und PDF Informationen
\usepackage[pdftex,plainpages=false,pdfpagelabels,
            pdftitle={title},
            pdfauthor={name}
            ]{hyperref}
%Farben für Links
%Farbige Ränder bei false und farbige Texte bei true
\hypersetup{colorlinks=true,citecolor=black,filecolor=black,linkcolor=black,urlcolor=black}

%Zwei Verzeichnisse für Inhalt und Anhang
\usepackage{appendix}
% \usepackage{minitoc} 
% \nomtcrule
% \renewcommand{\mtctitle}{Anhangsverzeichnis}
% \setlength{\mtcindent}{0pt}



%% Absätze sollen kein Einzug haben, aber dafür Abstände. %%
\parindent 0pt
\parskip 12pt